\chapter{Extending CyTex to Alternate Data Sources}\label{ch-ext-cytex}
The CyTex transform has previously demonstrated tremendous results on a number of benchmark datasets within the SER domain \cite{CyTexRef}. However, what is currently unknown, is the performance of such a method in alternate classification domains. This transformation method would prove even more powerful should its applications expand into these other types of problems. For this reason, this study also seeks to attempt to successfully extend the CyTex transforms use cases into additional domains. This section details the underlying theory and experimental results of such alternate data extensions. I should be noted that the extensions explored are not comprehensive and the CyTex transform may have further applications not explored in this project.\\ \\
\textbf{***USE THIS RESOURCE \cite{won2021music}!}


% =================================================
\section{Alternate Audio Data Sources}


% =================================================
\section{Classifying Musical Instruments OR NOTES}
\subsection{Musical Instrument Signals}


\subsection{Monophonic and Polyphonic Signals}


% =================================================
\section{Benchmark Musical Databases and Methods}


% =================================================
\section{CyTex in a Musical Context}
\subsection{CyTex Suitability}
As musical instruments produce acoustic signals, which typically exist within the human range of hearing, they are naturally similar to speech signals produced by humans. For this reason it is hypothesised that some meaningful patterns should be identifiable by a CyTex transformed deep learning process. 

\subsection{CyTex Classification Results}

